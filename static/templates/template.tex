% Vorlage für LaTeX-Abgaben / Gedächtnisprotokolle
% Erstellt von Jules Kreuer / not_a_feature
% Veröffentlicht am 15.12.2021 auf  ppi.fsi.uni-tuebingen.de
% ---
% !TeX TXS-program:compile = txs:///pdflatex/[--shell-escape]

\documentclass[a4paper]{article}

% Packages
% ---
\usepackage{amsmath} % Advanced Math Typesetting
\usepackage[utf8]{inputenc} % Unicode support (Umlaute etc.)
\usepackage[ngerman]{babel} % Change hyphenation rules

\usepackage{amssymb}
\usepackage{hyperref} % Links
\usepackage{graphicx} % Vilder
\usepackage{minted} % Source code highlighting
\usepackage[inline]{enumitem}
\usepackage{fullpage} % weniger abstand zu den seiten
\usepackage{tabularx} % bessere Tabellen
\usepackage{pdfpages} % pdf einbinden mit \includepdf[pages={2}]{x.pdf}

%Code
\usepackage{color}
\usepackage{colortbl}
\usepackage{textcomp}


\begin{document}
	% Name des Prüfenden
	\author{bei Prof. Musterfrau\\}
	% Name der Vorlesung
	\title{\vspace{-2cm}Gedächtnisprotokoll\\Name der Vorlesung}
	\date{\today{}} 
	\maketitle{} % Generates title
	\vspace{-1cm}
	% Hier bitte alle vorhandenen Informationen eintragen.
	\begin{minted}{text}
	Klausur:       Haupt/Nachklausur, SS/WS, JAHR
	Pruefer:       Prof. Pruefername
	Datum:         Datum der Pruefung
	Zeit:          Zeit zum Schreiben
	Punkte:	Anzahl der Punkte
	Hilfsmittel:   Erlaubte Hilfsmittel
		       Bsp: Taschenrechner, Cheat-Sheet
	Sprache:       Erlaubte Sprachen
	Modul:	 ggf. Modulnummer bei uneindeutigen Namen
	\end{minted}
	
	\section{Titel erste Aufgabe - n Punkte}
	\subsection{}
	Lorem ipsum dolor sit amet, consetetur sadipscing elitr,sed diam nonumy eirmod tempor invidunt ut labore et dolore magna aliquyam erat, sed diam voluptua.
	At vero eos et accusam et justo duo dolores et ea rebum.

	\subsection{}
	Lorem ipsum dolor sit amet, consetetur sadipscing elitr,sed diam nonumy eirmod tempor invidunt ut labore et dolore magna aliquyam erat, sed diam voluptua.
	At vero eos et accusam et justo duo dolores et ea rebum.
	\subsection{}
	\begin{minted}{python}
for i in range(0,10,2)
	print(i)
	\end{minted}
\end{document}
